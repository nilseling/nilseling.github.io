%----------------------------------------------------------------------------------------
%	PACKAGES AND OTHER DOCUMENT CONFIGURATIONS
%----------------------------------------------------------------------------------------

\documentclass[11pt,a4paper,sans]{moderncv} % Font sizes: 10, 11, or 12; paper sizes: a4paper, letterpaper, a5paper, legalpaper, executivepaper or landscape; font families: sans or roman

\moderncvstyle{casual} % CV theme - options include: 'casual' (default), 'classic', 'oldstyle' and 'banking'
\moderncvcolor{blue} % CV color - options include: 'blue' (default), 'orange', 'green', 'red', 'purple', 'grey' and 'black'


\usepackage[scale=0.75]{geometry} % Reduce document margins
%\setlength{\hintscolumnwidth}{3cm} % Uncomment to change the width of the dates column
%\setlength{\makecvtitlenamewidth}{10cm} % For the 'classic' style, uncomment to adjust the width of the space allocated to your name

%----------------------------------------------------------------------------------------
%	NAME AND CONTACT INFORMATION SECTION
%----------------------------------------------------------------------------------------

\firstname{Nils} % Your first name
\familyname{Eling} % Your last name

% All information in this block is optional, comment out any lines you don't need
\address{Widumweg 7, 8049 Zurich, Switzerland}
\mobile{+41-762982267}
\email{nils.eling@dqbm.uzh.ch}


%----------------------------------------------------------------------------------------

\begin{document}

\makecvtitle % Print the CV title

%----------------------------------------------------------------------------------------
%	EDUCATION SECTION
%----------------------------------------------------------------------------------------

\section{Employment}
\cventry{2023--present}{Senior computational scientist}{DQBM/IMHS}{\textit{University of Zurich/ETH Zurich}}{}{}
\cventry{2019--2022}{Postdoctoral fellow}{DQBM}{\textit{University of Zurich}}{}{}

\section{Education}
\cventry{2015--2019}{Computational Biology, PhD}{EMBL-EBI}{\textit{University of Cambridge}}{}{}
\cventry{2012--2015}{Molecular Biotechnology, M.Sc.}{University of Heidelberg}{\textit{Grade: 1.0}}{}{}
\cventry{2009--2012}{Molecular Biotechnology, B.Sc.}{University of Heidelberg}{}{\textit{Grade: 1.1}}{}  % Arguments not required can be left empty

\section{Projects}

\cvitem{\textbf{Senior scientist}}{\emph{Computational approaches for highly multiplexed image analysis.}}
\cvitem{Supervisor}{Bernd Bodenmiller, Single cell systems biology of cancer, DQBM/IMHS, University of Zurich/ETH Zurich}
\cvitem{Description}{My role involves the development of computational approaches for multiplexed image analysis, teaching and training, and data analysis as part of the IMMUcan project.}

\cvitem{\textbf{Postdoc}}{\emph{Profiling the emergence of cellular heterogeneity in breast cancer organoids.}}
\cvitem{Supervisor}{Bernd Bodenmiller, Single cell systems biology of cancer, DQBM, University of Zurich}
\cvitem{Description}{My postodoctoral work focuses on understanding the emergence of phenotypic heterogeneity in breast cancer and how this relates to treatment efficiency. In parallel, I develop computational approaches and tools for analysing multiplexed imaging data.}

\cvitem{\textbf{PhD}}{\emph{Quantifying expression variability
 in single-cell RNA sequencing data.}}
\cvitem{Supervisor}{John Marioni, Single-cell and computational biology, EMBL-EBI and CRUK CI, Cambridge}
\cvitem{Description}{During my PhD I focused on quantifying and understanding the functional role of transcriptional variability in immune responses and development. For this, I developed a statistical approach to correct the confounding effect of mean expression on transcriptional variability.}

\vspace{2ex}

\cvitem{\textbf{M.Sc.}}{\emph{Characterization of programmed cell death modalities induced by piperlongumine
and artesunate in pancreatic cancer cells.}}
\cvitem{Supervisor}{Dr. Anne Hamacher-Brady, Lysosomal Systems Biology, DKFZ, Heidelberg}

\vspace{2ex}

\cvitem{\textbf{B.Sc.}}{\emph{Modelling the Nrf2-Keap1 signalling pathway in human pancreatic carcinoma cells.}}
\cvitem{Supervisor}{Dr. Nathan Brady, Systems Biology of cell death mechanisms, DKFZ, Heidelberg}

%----------------------------------------------------------------------------------------
%	WORK EXPERIENCE SECTION
%----------------------------------------------------------------------------------------

\section{Research Experience}

\subsection{Internships}

\cventry{2013--2014}{Research Internship}{\textsc{The Garvan Institute of Medical Research}}{Sydney, Australia}{}{Defining the role of Sirtuin 1 in the onset of Pancreatic Ductal Adenocarcinoma.}

\cventry{2012}{Research Internship}{\textsc{The Scripps Research Institute}}{La Jolla, CA}{}{Activation of CD8$^+$ T cells \emph{in vitro} as well as \emph{in vivo} in order to specifically target pancreatic tumors in 8--14 week old mice.
}
%------------------------------------------------
\cventry{2011}{Industrial Internship}{\textsc{Merck KGaA}}{Darmstadt, Germany}{}{Proliferation induction in human cancer stem cells using different cytokines.
}

\cventry{2009}{Research Internship}{\textsc{University of Duisburg-Essen}}{Duisburg, Germany}{}{Collaboration with the SulfoSYS project in order to analyse the central carbohydrate metabolism of \emph{S. solfataricus}.
}

\cventry{2005}{High School Intern}{\textsc{Evonik Goldschmidt GmbH}}{Essen, Germany}{}{Characterisation of polyurethane foam properties.}


%------------------------------------------------

\subsection{Research Assistances}

\cventry{2012--2014}{Research Assistant}{Max Planck Institute for Medical Research}{}{}{\textsc{Django}/\textsc{MySQL} based website development to process spatially annotated electron imaging data.}

\cventry{2011--2012}{Research Assistant}{Complex biological systems group}{IWR, Heidelberg}{}{ODE based modelling of the chemotactic pathway of \emph{E. coli}.}

\cventry{2010--2011}{Research Assistant}{Signal transduction in cancer and metabolism}{DKFZ, Heidelberg}{}{Using \emph{D. melanogaster} as model organism for analysing caloric restriction and the Akt/mTOR signalling pathway.}


%----------------------------------------------------------------------------------------
%	PUBLICATION SECTION
%----------------------------------------------------------------------------------------

\section{Selected publications}

\cventry{2021}{\textnormal{\textit{An end-to-end workflow for multiplexed image processing and analysis}}}{\textnormal{Windhager, J.*, Bodenmiller, B., \underline{Eling, N.*}}}{\textit{bioRxiv}}{*Corresponding author}{}

\cventry{2020}{\textnormal{\textit{cytomapper: an R/Bioconductor package for visualisation of highly multiplexed imaging data}}}{\textnormal{\underline{Eling, N.*}, Damond, N., Hoch, T., Bodenmiller, B.}}{\textit{Bioinformatics}}{*Corresponding author}{}

\cventry{2019}{\textnormal{\textit{Challenges in measuring and understanding biological noise}}}{\textnormal{\underline{Eling, N.}, Michael Morgan, John Marioni}}{\textit{Nature Reviews Genetics}}{}{}

\cventry{2019}{\textnormal{\textit{Staged developmental mapping and X chromosome transcriptional dynamics during mouse spermatogenesis}}}{\textnormal{Ernst, C.*, \underline{Eling, N.}* \textit{et al.}}}{\textit{Nature Communications}}{*Co-first authors}{}

\cventry{2018}{\textnormal{\textit{Correcting the mean-variance dependency for differential variability testing using single-cell RNA
sequencing data}}}{\textnormal{\underline{Eling, N.} \textit{et al.}}}{\textit{Cell Systems}}{}{}

\cventry{2018}{\textnormal{\textit{Whole-Body Single-Cell Sequencing Reveals Transcriptional Domains in the Annelid Larval Body}}}{\textnormal{Achim, K.*, \underline{Eling, N.}* \textit{et al.}}}{\textit{Molecular Biology and Evolution}}{*Co-first authors}{}

\cventry{2017}{\textnormal{\textit{Aging increases cell-to-cell transcriptional variability upon immune stimulation}}}{\textnormal{Martinez-Jimenez, C.P.*, \underline{Eling, N.}* \textit{et al.}}}{\textit{Science}}{*Co-first authors}{}

\cventry{2015}{\textnormal{\textit{Identification of artesunate as a specific activator of ferroptosis in pancreatic cancer cells}}}{\textnormal{\underline{Eling, N.} \textit{et al.}}}{\textit{Oncoscience}}{2(5), 517-532}{}


%----------------------------------------------------------------------------------------
%	AWARDS SECTION
%----------------------------------------------------------------------------------------

\section{Scholarships and awards}

\cvitem{2021-2022}{Marie Skłodowska-Curie Actions Individual Fellowship}
\cvitem{2019-2020}{EMBO Long-Term Fellowship}
\cvitem{2017}{Kurt Hahn Award for German nationals in Cambridge}
\cvitem{2015-2019}{EMBL international PhD fellowship}
\cvitem{2011-2015}{Scholar of the foundation of German business}
\cvitem{2011-2015}{Scholar of e-fellows.net}

%----------------------------------------------------------------------------------------
%	CONFERENCE SECTION
%----------------------------------------------------------------------------------------

\section{Conferences and workshops}

\subsection{Talk}
\cvitem{2022}{ISSCR Spatial Transcriptomics (invited)}
\cvitem{2022}{Cytométrie de Masse, 4$^e$ édition (invited)}
\cvitem{2022}{Centre for Computational Biomedicine, Harvard (invited)}
\cvitem{2021}{Frontline Genomics, Single Cell \& Spatial Omics ONLINE (invited)}
\cvitem{2021}{University of Sydney, Statistical Bioinformatics Seminar Series (invited)}
\cvitem{2018}{Francis Crick institute artificial intelligence seminar (invited)}
\cvitem{2018}{EBI Sanger Cambridge PhD Symposium}
\cvitem{2017}{EMBL Lab Day}
\cvitem{2015}{EMBL PhD Symposium}

\subsection{Poster}
\cvitem{2022}{Applied Bioinformatics in Life Sciences}
\cvitem{2021}{AACR}
\cvitem{2020}{Systems biology of cancer: promises of artificial intelligence}
\cvitem{2020}{BioC 2020}
\cvitem{2015-2017}{Single Cell Genomics}
\cvitem{2016}{Single Cell Biology}
\cvitem{2016}{Quantitative Genomics}

\subsection{Workshop}
\cvitem{2022}{EuroBioC 2022 (presenter)}
\cvitem{2022}{BioC 2022 (presenter)}
\cvitem{2021}{EMBO Lab Leadership (attendee)}
\cvitem{2021}{BioC 2021 (presenter)}
\cvitem{2021}{Indiana O’Brien Center Microscopy Workshop (invited presenter)}
\cvitem{2016-2017}{Academy for PhD Training in Statistics (attendee)}
\cvitem{2015}{Statistics and Computing in Genome Data Science (attendee)}

\subsection{Conference/meeting organiser}
\cvitem{2023}{Highly Multiplexed Tissue Imaging Computational Workshop}
\cvitem{2023}{Highly Multiplexed Imaging Developers Meeting}
\cvitem{2018}{Quantitative Genomics}
\cvitem{2017}{Science and Society: Gut feeling}
\cvitem{2016}{Science and Society: Rewriting the Code of Life}
\cvitem{2015}{EBI Sanger Cambridge PhD Symposium}

\subsection{Hackathon}
\cvitem{2020}{Hack Zurich}
\cvitem{2017}{Human Cell Atlas}
\cvitem{2017}{MLH Prime}
\cvitem{2017}{Hack Cambridge Recurse}

\section{Teaching and supervision}

\subsection{Teaching}
\cvitem{2023}{Highly Multiplexed Tissue Imaging Computational Workshop}
\cvitem{2022}{ETH/UZH PhD Program in Cancer Biology Module B - Multiplexed image analysis}
\cvitem{2020}{DQBM online course: Introduction to data analysis}
\cvitem{2016}{EMBL: Bioinformatics Teaching Module}
\cvitem{2015}{Machine Learning for Personalised Medicine summer school (assistant)}

\subsection{Supervision}
\cvitem{Since 2022}{Computational research assistant}
\cvitem{2022}{Computational Master student (co-supervision)}
\cvitem{2022}{Computational rotation student}
\cvitem{2021}{Experimental rotation student (SEMP and PROMOS awardee)}
\cvitem{2020}{Computational Master student}
\cvitem{2020}{Experimental Master student (co-supervisor)}
\cvitem{2020}{Computational rotation student}

\section{Engagement}

\subsection{Societies}
\cvitem{2021}{DQBM JUSCOR}

\subsection{Scientific reviewer}
\cvitem{2022}{Bioconductor}
\cvitem{2021}{Bioinformatics}

%----------------------------------------------------------------------------------------
%	COMPUTER SKILLS SECTION
%----------------------------------------------------------------------------------------

\section{Technical skills}

\cvitem{Basic}{Matlab, C++}
\cvitem{Intermediate}{Python, html/css, JavaScript}
\cvitem{Advanced}{R, git, \LaTeX, bash, Docker}


%----------------------------------------------------------------------------------------
%	LANGUAGES SECTION
%----------------------------------------------------------------------------------------

\section{Languages}

\cvitemwithcomment{German}{Mother tongue}{}
\cvitemwithcomment{English}{Advanced}{Conversationally and scientifically fluent}
\cvitemwithcomment{French}{Basic}{Basic words and phrases}

\end{document}